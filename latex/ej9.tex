\section{Ejercicio 9}
En esta parte del trabajo vamos a analizar la \textit{fairness} del algoritmo de scheduling
\texttt{SchedLottery}. Tomamos como definici�n de \textit{fairness} al hecho de que cada
proceso reciba igual cantidad de tiempo de CPU, o m�s precisamente, un tiempo apropiado para
cada proceso de acuerdo a su prioridad y carga de trabajo.

A continuaci�n vamos a mostrar los experimentos realizados para poder ver que efectivamente
cuando se realizan \textit{n} experimentos, el scheduler es realmente justo cuando \textit{n} aumenta su tama�o.
Notar que es necesario hacer m�s de una corrida ya que debido al factor pseudoaleatorio del
algoritmo, realizar una o dos corridas no es suficiente para ver que efectivamente el scheduler
es justo.