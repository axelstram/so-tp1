\subsection{Introducci�n}
En esta secci�n se extiende el simulador con un nuevo algoritmo de scheduling, \textit{Round Robin}, y se lo testea con 
diversos lotes de tareas.

\subsection{Ejercicio 3}
El objetivo de este ejercicio es implementar la pol�tica de scheduling \textit{Round Robin}. La funci�n m�s importante es \texttt{tick(cpu, motivo)}, cuya implementaci�n se 
describe a continuaci�n: si el motivo es \textbf{TICK} o \textbf{BLOCK}, se aumenta el contador de ticks del core correspondiente. Si este contador supera el quantum del core, se 
vuelve a poner el contador en $0$, se encola la tarea actual y comienza a ejecutarse la siguiente tarea en la cola; caso contrario, se sigue ejecutando la tarea actual.

Si el motivo es \textbf{EXIT}, sencillamente se devuelve la proxima tarea en la cola, sin encolar nuevamente la tarea actual. En caso de no haber m�s tareas, se ejecuta \textbf{IDLE\_TASK}.

Para m�s detalles, consultar la implementaci�n en el archivo \textit{sched\_rr.cpp}.

\subsection{Ejercicio 4}

\subsection{Ejercicio 5}