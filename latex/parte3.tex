\subsection{Introducci�n}
En esta secci�n se evaluan las pol��ticas de scheduling implementadas, utilizando diversas m�tricas especificadas m�s adelante.

\subsection{Ejercicio 6}
El objetivo de este ejercicio es programar un tipo de tarea \textbf{TaskBatch}, que durante \textit{total\_cpu} ciclos, realize \textit{cant\_bloqueos} llamadas bloqueantes, 
en momentos elegidos pseudoaleatoriamente. La implementaci�n es bastante directa, con la semilla del generador de n�meros pseudoaleatorios inicializada con la fecha del sistema
al momento de ejecutar la funci�n. Las llamadas bloqueantes se lanzan si \textit{rand()} devuelve un n�mero impar. 

Para m�s detalles, consultar la implementaci�n en \textit{tasks.cpp}.

\subsection{Ejercicio 7}

En este ejercicio debemos elegir 2 m�tricas diferentes y testear un lote de tareas \textbf{TaskBatch}, todas ellas con igual uso de CPU pero con diversas
cantidades de bloqueos. El lote de tareas utilizado es el \textit{lote3.tsk}.

Las m�tricas que elegimos fueron:
\begin{itemize}
 \item Turnaround
 \item Waiting Time 
\end{itemize}

Definidas en [Sil1] como:
\newline

Turnaround: 

Waiting Time:


\subsection{Ejercicio 8}

\subsection{Ejercicio 9}

\subsection{Ejercicio 10}