\subsection{Introducci�n}
En esta secci�n se evaluan las pol��ticas de scheduling implementadas, utilizando diversas m�tricas especificadas m�s adelante.

\subsection{Ejercicio 6}
El objetivo de este ejercicio es programar un tipo de tarea \textbf{TaskBatch}, que durante \textit{total\_cpu} ciclos, realize \textit{cant\_bloqueos} llamadas bloqueantes, 
en momentos elegidos pseudoaleatoriamente. La implementaci�n es bastante directa, con la semilla del generador de n�meros pseudoaleatorios inicializada con la fecha del sistema
al momento de ejecutar la funci�n. Las llamadas bloqueantes se lanzan si \textit{rand()} devuelve un n�mero impar. 

Para m�s detalles, consultar la implementaci�n en \textit{tasks.cpp}.

\subsection{Ejercicio 7}

\subsection{Ejercicio 8}

\subsection{Ejercicio 9}

\subsection{Ejercicio 10}